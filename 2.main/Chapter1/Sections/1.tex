\section{مقدمه}
% AI + Physics
% Entanglement detection importance, Problem statement


%%%%%%%%%%%
در سال های اخیر، شاهد کاربرد های هوش مصنوعی در زمینه های مختلف علوم بوده ایم که منجر به حل مسائل بسیار پر کاربردی در علم و فناوری شده است. از این دست می توان به کاربرد هوش مصنوعی در علوم پایه مانند فیزیک، بیولوژی و شیمی اشاره کرد
\cite{Taddeo2018}.
با پیشرفت شگرف هوش مصنوعی و به وجود آمدن روش هایی مانند یادگیری ماشینی، قدرت حل مسائل پیچیده تر با آن نیز افزایش یافته است
\cite{chen2019}.

%%%%%%%%%%%%
درهم تنیدگی یکی از منابع کلیدی علوم اطلاعات کوانتومی است که کاربرد های بسیاری در رمزنگاری کوانتمی، تلپورت کوانتمی، و میکروسکوپی کوانتمی دارد.

در نتیجه شناسایی حالت های درهم تنیده برای طیف گسترده ای از فناوری ها و پدیده های کوانتومی ضروری است. این مساله که به عنوان مشکل جداپذیری نیز شناخته می شود
\cite{horodecki_1996}،
از مسائل معروف در زمینه ی اطلاعات کوانتمی است که حل آن به روش های تحلیلی با استفاده از کامپیوتر می تواند بسیار زمان بر و عملا غیر ممکن باشد
\cite{gharibia_2010}.
در این پروژه با استفاده از مفاهیم و ابزار های هوش مصنوعی به بررسی این مسئله پرداخته شده است تا بتوان با استفاده از ابزار های یادگیری ماشینی، حالت های درهم تنیده را تشخیص دهیم.
 از این رو، ضروری است تا در ابتدا مفاهیم مورد نیاز در هر دو زمینه، به اختصار توضیح داده شوند.
