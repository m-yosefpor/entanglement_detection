\section{مساله جداپذیری}

در بخش قبل، تعریف سیستم جداپذیر و درهم تنیده را گفتیم. حال سوال اینجاست که چگونه تشخیص دهیم یک سیستم مشخص، درهم تنیده یا جداپذیر است.
به عنوان مثال، برای یک سیستم دو بخشی با حالت خالص،‌ می توان با تجزیه اشمیت
\LTRfootnote{Schmidt decomposition}
به راحتی فهمید که سیستم درهم تنیده یا جداپذیر است.
اگر در این تجزیه، فقط یکی از ضرایب غیر صفر بود، حالت جداپذیر است، و در غیر این صورت حالت قطعا درهم تنیده است
\cite{Ekert1995,Sciara2017}.
با این حال، برای حالت های مخلط، نمایش تجزیه ی معادله ی
\ref{eqn:rho_sep}
ممکن است یکتا نباشد
\cite{Ekert1995,Sciara2017}.
بنابرین، اگر تجزیه ای به این صورت پیدا شود که ضرب حالت های زیر سیستم ها باشد ولی شرط
$p_i \geq 0$
یا
$\sum_i p_i =1$
برقرار نشود، نمی توان نتیجه گرفت که حالت جداپذیر نیست. چون ممکن است نمایش دیگری از این تجزیه وجود داشته باشد که شرط های معادله ی
\ref{eqn:rho_sep}
را برقرار سازد.
به همین خاطر، در حالت کلی، پیدا کردن اینکه یک ماتریس چگالی درهم تنیده است یا جداپذیر از نظر محاسباتی مساله سختی است
\cite{Vedral1997}،
اگرچه در بعضی حالت های خاص،‌ روش های ساده تحلیلی وجود دارد
\cite{horodecki_1996}.
به این مساله در علوم اطلاعات کوانتمی، مساله ی جداپذیری
\LTRfootnote{Separability Problem}
گفته میشود،
و در حالت کلی، نشان داده شده است که از نظر محاسباتی یک مساله
\lr{NP-Hard}
است
\cite{gharibia_2010}.
