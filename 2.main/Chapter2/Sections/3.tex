\section{انواع یادگیری ماشین}

در این بخش به سه دسته کلی روش های یادگیری ماشینی با توجه به برچسب گذاری داده ها می پردازیم.

\subsection{یادگیری نظارت شده}
اگر دسته بندی داده های آزمایش، از قبل به نوعی مشخص شده باشد، یا به عبارتی همه ی آن ها برچسب گذاری شده باشند، و الگوریتم ماشین از این برچسب ها برای یادگیری استفاده کنند، به این روش یادگیری نظارت شده می گویند. تعدادی از این روش ها که در این پایان نامه استفاده شده اند عبارتند از:
\lr{Ridge, \mbox{Random Forest}, \mbox{Support Vector}, Bagging, \mbox{Gradient Boosting}}
و غیره. همچنین دسته ای از شبکه های عصبی پیشخور
\LTRfootnote{Feedforward Neural Network}
\lr{(FFNN)}،
مانند شبکه های عصبی پیچشی
\LTRfootnote{Convolutional Neural Network}
\lr{(CNN)}
و
شبکه های عصبی بازگشتی
\LTRfootnote{Recursive Neural Network}
\lr{(RNN)}
نیز از دسته ی یادگیری نظارت شده هستند.

\subsection{یادگیری نظارت نشده}
اگر الگوریتم برای دسته بندی، فقط از ویژگی های هر نمونه استفاده کند و استفاده ای از برچسب داده ها نکند، به آن ها یادگیری نظارت نشده می گویند. تعدادی از این روش ها که در این پایان نامه استفاده شده اند عبارتند از:
\lr{OPTICS},
\lr{K-means},
\lr{DBSCAN},
\lr{Birch}.
همچنین از روش های شبکه عصبی نظارت نشده می توان به
\lr{Boltzman Machine}
ها اشاره کرد، که دسته ای مهم از روش های یادگیری نظارت نشده هستند.

\subsection{یادگیری شبه نظارت شده}
دسته ای از روش ها هم هستند که ترکیب این دو اند. یعنی قسمتی از داده ها برچسب گذاری، ‌و قسمت دیگری برچسب گذاری نشده اند.
این روش ها با استفاده از برچسب های موجود، سعی در بهتر کردن نتیجه دارند. روش های شبه نظارت شده ی مورد استفاده در این پایان نامه عبارتند از
\lr{LabelPropagation}،
\lr{LabelSpreading}،
و
\lr{\mbox{Anomaly Detection}}.
در ادامه مبانی تعدادی از این روش ها به اختصار توضیح داده می شود.
