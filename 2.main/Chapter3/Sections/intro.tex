\section{مقدمه}

با توجه به پیش نیاز های گفته شده در فصل قبل، حال آماده ایم تا به مسئله ی اصلی بپردازیم. در ادامه، ابتدا به نحوه تولید داده ها و برخی بررسی های آماری انجام گرفته روی آن ها می پردازیم.
سپس مدل های یادگیری ماشینی کلاسیکی را آموزش می دهیم و نتایج آن ها و زمان یادگیری هر یک را با یکدیگر مقایسه می کنیم.
در ادامه با استفاده از روش شبکه های عصبی و شبکه های خودرمزگذار مدل را با تعداد کمی اندازه گیری آموزش می دهیم. سپس با استفاده از تقارن استوانه ای به بررسی مجدد موضوع می پردازیم.
همچنین اثر نوفه بر این داده ها را بررسی کرده ایم.
در آخر روش تشحیض ناهنجاری
\LTRfootnote{Anomaly Detection}
که از روش های یادگیری نظارت نشده است را بر روی داده ها بررسی می کنیم.
