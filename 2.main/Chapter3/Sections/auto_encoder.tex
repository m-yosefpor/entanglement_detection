\section{شبکه عصبی خود رمزگذار}

% non-linear is hard to measure

در قسمت قبل، توانستیم با داشتن ۱۵ اندازه گیری، به طور تقریبا کامل حالت ها را از یکدیگر تفکیک کنیم. در این قسمت، معماری دیگری بر اساس شبکه های عصبی خود رمزگذار ارایه می دهیم که بتواند با تعداد کمتری اندازه گیری این کار را انجام دهد. معماری ارایه شده در شکل
\ref{fig:nn_nonlin}
آورده شده است. در بخش میانی این شبکه، یک گلوگاه دو گره ای ایجاد شده است، تا همه ی اطلاعات مورد نیاز برای تشخیص در هم تنیدگی، از این گلوگاه عبور کند. این شبکه را مجبور کرده که بخش اول که به عنوان کدگذار شناخته می شود، یعنی از لایه ی اول تا گلوگاه، و بخش دوم که به عنوان کدگشا شناخته میشود، یعنی از گلوگاه تا خروجی را به صورت بهینه برای تفکیک حالت ها تنظیم کند. با توجه به اینکه دو گره میانی، هر یک تابع خطی ویژگی های ورودی هستند،‌ پس در واقع هر یک بیانگر آزمایش غیر خطی از ضرب ماتریس های پاولی هستند. پس از آنکه مدل به طور کامل آموزش یافت،‌ می توان تابع های مورد نظر را با توجه به بخش اول شبکه به دست آورد. همچنین با داشتن مقادیر این اندازه گیری ها، می توان از بخش کدگشای شبکه استفاده کرد و نتیجه ی درهم تنیده بودن یا جداپذیر بودن حالت را به دست آورد.

\input{\figurePATH{3}{nn_nonlin}}

در این حالت شبکه موفق شد به صحت
۹۸٪
در تفکیک حالت ها دست پیدا کند.
برای اطلاعات دقیق تر، ماتریس درهم ریختگی در جدول
\ref{tab:confusion_matrix}
آورده شده است.
\input{\tablePATH{3}{confusion_matrix}}
