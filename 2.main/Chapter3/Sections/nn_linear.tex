\section{شبکه عصبی با تعداد ویژگی کمتر}

در بخش قبل توانستیم با دو اندازه گیری غیر خطی، حالت های در هم تنیده از حالت های جداپذیر را تشخیص دهیم. اما با توجه به اینکه انداره گیری های غیر خطی در آزمایشگاه چالش بر انگیز هستند، در این بخش معماری دیگری را ارایه می دهیم که بتواند با تعداد کمتری اندازه گیری خطی بهینه این کار را تا حد قابل قبولی انجام دهد.

معماری شبکه عصبی ارایه شده در شکل
\ref{fig:nn_lin}
آورده شده است. همانطور که مشاهده می شود، در لایه ی اول، به هر تعداد آزمایش خطی که مورد نیاز است، گره می گذاریم. در این مثال سه گره آورده شده است. نکته قابل توجه دیگر این است که تابع فعلا سازی لایه ی اول تابع همانی در نظر گرفته شده است، و لایه های دوم به بعد فقط از تابع غیر خطی
\lr{relu}
استفاده می کنند. لذا مقادیر گره های لایه ی اول هر کدام یک ترکیب خطی از مقادیر ویژگی ها هستند. یعنی معماری شبکه، ماشین را مجبور می کند ترکیب خطی بهینه ای برای تشخیص حالت های در هم تنیده پیدا کند. با توجه به اینکه مقادیر ویژگی ها، خود مقدار چشم داشتی اندازه گیری های ضرب ماتریس های پاولی می باشند، پس مقدار گره های لایه ی اول نیز هر کدام معادل یک اندازه گیری ترکیب خطی از ضرب ماتریس های پاولی هستند. با به دست آمدن ضرایب یال های لایه ی اول می توان این ترکیب خطی، و در نتیجه عملگر مورد نیاز برای تشخیص در هم تنیدگی را به دست آید.

\input{\figurePATH{3}{nn_lin}}

این کار برای تعداد اندازه گیری های خطی متفاوت تکرار شد و نتایج در شکل
\ref{fig:acc_rec_n}
نمایش داده شده است.
همانگونه که انتظار می رود، با افزایش تعداد اندازه گیری ها، صحت مدل افزایش می یابد تا در نهایت با داشتن ۱۵ اندازه گیری، می تواند به طور تقریبا کامل حالت ها را تفکیک کند.
به طور مثال با داشتن فقط ۳ اندازه گیری خطی،‌ مدل می تواند صحت
تقریبا
۸۰٪
را به دست آورد. توجه کنید که با وجود مصالحه بین دقت و بازیابی، می توان آستانه، را به گونه ای تغییر داد که دقت تا مقدار دلخواه بالا برود. مثلا در صورت نیاز دقت
۹۵٪،
بازیابی مدل
۶۰٪
می شود. این بدین معنی است که اگر بخواهیم مدل هر حالتی را که درهم تنیده تشخیص دهد، به احتمال ۹۵ درصد واقعا درهم تنیده باشد، مدل می تواند
۶۰٪
حالت های در هم تنیده را تشخیص دهد. یا به عنوان مثال مهم دیگر، اگر بخواهیم با قطعیت تقریبا کامل حالت های درهم تنیده را تشخیص دهیم، می توانیم آستانه را طوری تنظیم کنیم که دقت بیشتر از
۹۹،۹۹۹٪
باشد.
در این صورت، هر حالت تشخیص داده شده توسط ماشین به عنوان حالت درهم تنیده، با قطعیت تقریبا کامل، واقعا درهم تنیده است. پس به نوعی ماشین به عنوان یک شاهد درهم تنیدگی عمل میکند. با تنظیم چنین آستانه ای، ماشین به بازیابی
۲۱٪
می رسد، که یعنی می تواند با قطعیت تقریبا کامل ۲۱٪ حالت های درهم تنیده را با فقط سه اندازه گیری خطی تشخیص دهد.

\input{\figurePATH{3}{acc_rec_n}}

نتیجه ی جالب دیگری که در این بخش به دست آمد، بررسی ضرایب لایه ی اول شبکه عصبی است. این ضرایب در واقع همان ضرایب ترکیب خطی عملگر مورد نظر از ضرب ماتریس های پاولی است. از طرف دیگر، اندازه ی ضریب هر یک از این عملگر ها، به معنی اهمیت بیشتر آن عملگر در تشخیص حالت های در هم تنیده است. نتیجه ی به دست آمده، با نتایج به دست آمده از بخش اهمیت ویژگی های روش های کلاسیکی کاملا تطابق دارد. همانطور که مشاهده می شود، مهم ترین ویژگی ها در این بخش نیز
ضرب تانسوری دو ماتریس پاولی یکسان بودند، یعنی
$\sigma_x \otimes \sigma_x , \sigma_y \otimes \sigma_y , \sigma_z \otimes \sigma_z$.

\input{\figurePATH{3}{first_layer}}
