\section{جمع بندی}
در این پایان نامه، قصد داشتیم تا حالت های درهم تنیده و جداپذیر در یک سیستم دو کیوبیتی را از یکدیگر با استفاده از ابزار های یادگیری ماشینی تفکیک کنیم. در فصل اول، به بیان مفاهیم مورد نیاز حوزه اطلاعات کوانتمی، و تبیین مساله به زبان ریاضی پرداختیم. در فصل دوم مفاهیم هوش مصنوعی و یادگیری ماشینی مورد استفاده در این پایان نامه به اختصار توضیح داده شد. سپس در فصل سوم، روش تولید داده ها، بررسی آماری آن ها، و مدل های یادگیری ماشینی ارایه شد و نتایج آن ها ارزیابی و مقایسه شد.


در ابتدا، سعی کردیم بتوانیم با داشتن توصیف کامل سیستم (نتایج ۱۵ اندازه گیری روی سیستم دو کیوبیت)، ماشین را آموزش دهیم تا بتواند حالت های جداپذیر و درهم تنیده را تفکیک کند و به معیار
\lr{PPT}
یا معیاری هم ارز آن دست پیدا کند.
ابتدا این کار را با استفاده ما از روش های کلاسیکی یادگیری ماشینی استفاده کردیم و نتایج آموزش و همچنین زمان آموزش و ارزیابی را در جدولی مرتب کردیم. مشاهده شد که بعضی روش ها مانند
\lr{Ridge}
توانستند به صورت تقریبا کامل
(
صحت
۹۹.۷٪
)
حالت ها را از دیگر تفکیک کنند
همچنین اهمیت هر یک از این اندازه گیری ها را در این روش های کلاسیکی مورد بررسی قرار دادیم و دیدیم بعضی از این اندازه گیری ها اهمیت بیشتری از بقیه در تعیین نتیجه ی درهم تنیده یا جداپذیر بودن دارند.
سپس با روش های شبکه ی عصبی این کار را مجددا تکرار کردیم و مشاهده شد که با داشتن توصیف کامل سیستم، آن ها نیز قادر هستند به تفکیک نسبتا کامل
(
صحت
۹۹.۹٪
)
برسند.


سپس مدل دیگری بر اساس شبکه های خود رمزگذارارایه کردیم که موفق شد با فقط دو اندازه گیری غیر خطی را به صحت
بالای
۹۹٪
برسند. با توجه به اینکه اندازه گیری های غیر خطی در آزمایشگاه چالش برانگیز و هزینه بر هستند، مدل دیگری ارایه دادیم که بتواند تعداد کمی اندازه گیری خطی بهینه برای تشخیص بخش قابل توجهی از حالت های در هم تنیده پیدا کند. به عنوان مثال توانستیم فقط با ۳ اندازه گیری خطی، به صحت
۸۰٪
برسیم. می توان با تغییر آستانه ماشین، دقت تشخیص حالت های درهم تنیده را تا حد مطلوب بالا برد. مثلا اگر دقت
۹۹.۹۹٪
مورد نیاز باشد، ماشین فراخوانی
۲۱٪
پیدا می کند. در این صورت ماشین به عنوان یک شاهد درهم تنیدگی عمل می کند.

با توجه به اینکه حالت های تولید شده در آزمایشگاه، معمولا داری تقارن های مشخصی هستند، ما مجددا مراحل قبل را برای حالت های با تقارن استوانه ای تکرار کردیم و نتایج بهتری کسب کردیم. در این حالت ماشین فقط با دو اندازه گیری خطی صحت
۶۰٪
و با ۴ اندازه گیری خطی
به صحت
۹۸٪
رسید. همچنین با توجه به اینکه نتایج اندازه گیری های انجام شده در آزمایشگاه همیشه همراه با نویز هستند، ما مدل های قبلی را با وجود اثر نویز جمع شونده گاوسی بررسی کردیم و مشاهده شد که مدل ها پایداری مطلوبی در صورت وجود نویز دارند.

در ادامه فصل ۳، از چندین روش های یادگیری نظارت نشده و شبه نظارت شده هم استفاده کرده و آن مدل ها را ارزیابی کردیم. در این مدل ها، روش
\lr{OPTICS}
از روش های نظارت نشده و روش
\lr{Anomaly Detecting}
از روش شبه نظارت شده، بهترین نتیاج را حاصل کردند و به ترتیب صحت
۷۷٪
و
۸۱٪
دارند.
هرچند صحت این دسته از مدل ها، از حالت نظارت شده کمتر است، اما قابلیت تعمیم به ابعاد بالاتر، برای این مدل ها راحت تر است.

تمامی مدل های ارایه شده در این پایان نامه، به عنوان اثبات ایده ی استفاده از شبکه های عصبی و شبکه های خود رمزگذار در پیدا کردن اندازه گیری های بهینه برای تشخیص درهم تنیدگی هستند، و باید در نظر داشت که پژوهش ما محدودیت منابع محاسباتی مشخصی داشت. صحت هر یکی از این مدل ها، می تواند با تنظیم فراپارامتر های شبکه و افزایش تعداد داده های آموزش، و لایه های شبکه بهبود یابد. مشخص نیست تا چه میزان می توان این مدل ها را بهبود داد و تا چه حد صحت هر یک را افزایش داد.
