\section{شبکه های خود رمزگذار}

شبکه های خود رمز گذار، نوعی از شبکه های عصبی هستند که به طرز بهینه ای داده ها را کدگذاری می کنند و بعد فضا را کاهش می دهند
\cite{pmlr-v27-baldi12a}.
معماری این شبکه ها به این صورت است که ابتدا تعداد گره های در لایه های اول زیاد و در لایه های بعدی کم می شود تا اینکه به تعداد کمی گره ی مطلوب برسد. سپس مجددا تعداد گره ها شروع به افزایش می کند تا به تعداد گره ی مطلوب برسد. در شکل
\ref{fig:autoencoder_arch}
\LTRfootnote{\texttt{$https://laptrinhx.com/autoencoders-3518090441/$}}
معماری این شبکه به تصویر کشیده شده است.
در این صورت لایه ی میانی که دارای کمترین گره است، به صورت یک گلوگاه برای جریان اطلاعات عمل می کند. در نتیجه،‌ تمامی اطلاعاتی که ورودی را به خروجی تبدیل می کند،‌ مجبور است از گلوگاه عبور کند و به عبارتی در این تعداد کم گره ها کد شود. به قسمت اول این شبکه که اطلاعات ورودی را در تعداد کم گره ی گلوگاه کد می کند، کدگذار، و به قسمت دوم این شبکه که با استفاده از مقادیر گره های گلوگاه، خروجی را بازسازی می کند،‌ کدگشا گفته می شود.
