\begin{figure}
    \centering
    \begin{subfigure}{.5\textwidth}
        \centering
        \includegraphics[width=0.9\linewidth]{\pdfPATH{1}{ent}}
        \caption{شاهد در هم تنیدگی خطی و غیر خطی.}
        \label{fig:witness}
    \end{subfigure}%
    \begin{subfigure}{.5\textwidth}
        \centering
        \includegraphics[width=0.9\linewidth]{\pdfPATH{1}{PPT2}}
        \caption{شاهد درهم تنیدگی لازم و کافی.}
        \label{fig:iff_witness}
    \end{subfigure}
    \caption{  شاهد درهم تنیدگی فضای حالت ها را به دو زیر مجموعه افراز میکند و  می تواند بخشی یا همه ی حالت های در هم تنیده را تشخیص دهد. شاهد درهم تندیگی می تواند خطی یا غیر خطی باشد (شکل آ). معیار پرز-هورودکی یک شاهد لازم و کافی برای سیستم دو کیبوبیت است. این شاهد فضای حالت ها را به طور کامل به دو بخش فقط درهم تنیده و فقط جداپذیر افراز می کند (شکل ب).}
    \label{fig:witnesses}
\end{figure}

