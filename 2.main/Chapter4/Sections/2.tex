\section{پیشنهادات}

در ادامه، پیشنهاداتی برای ادامه ی کار مطرح می شود.
%%%%%%%%%%%%%%%%%%%%%%%%%%%%%%%%%%%%%%%%%%%%%%%%%%%%%%%%%%%%%%
\subsection{بررسی تقارن های دیگر}
یکی از پیشنهاد های موجود برای ادامه ی کار، این است که تقارن های دیگر مانند تقارن کروی یا تقارن های آینه ای نیز بررسی شود. برای این کار لازم است تا شروط لازم این تقارن ها به صورت معادله هایی برای ویژگی های ماتریس چگالی به دست آید، و سپس مشابه الگوریتم ارایه شده در کد، ابتدا عملگر مورد نیاز برای متقارن سازی حالت ها پیدا شده، و روی حالت های تولید شده اثر داده شود که حالت هایی با تقارن مورد نظر ساخته شود. سپس می توان مدل های ارایه شده در این پایان نامه را مجددا روی آن تقارن ها نیز بررسی کرد.
%%%%%%%%%%%%%%%%%%%%%%%%%%%%%%%%%%%%%%%%%%%%%%%%%%%%%%%%%%%%%%
\subsection{ابعاد بالاتر}
در این پایان نامه، به سیستم دو کیوبیت پرداخته شد،‌اما اکثر روش های ارایه شده در این پایان نامه، قابل تعمیم به ابعاد بالاتر هستند. برای این کار لازم است که شاهد های متفاوتی بر داده های تولید شده اعمال شوند تا هر یک بخشی از این حالت ها را به صورت درهم تنیده تشخیص دهند. حالت های خالص را نیز می توان با تجزیه ی اشمیت برچسب گذاری کرد. همچنین با روش ارایه شده در فصل سوم، می توان ماتریس های چگالی جداپذیر تولید کرد. یک چالش دیگر ابعداد بالاتر، مشقت بعدچندی
\LTRfootnote{Curse of Dimensionality}
است. با افزایش ابعاد زیر فضا ها، بعد سیستم مرکب دوبخشی به صورت
$O(N^2)$
سریعا رشد می کند و داده های موجود برای یادگیری در فضا بسیار تنک و پراکنده می شوند. در نتیجه به تعداد بسیار بیشتری داده برای یادگیری نیاز است. علاوه بر این، ویژگی های هر ماتریس چگالی نیز به صورت
به شدت افزایش می یابد و این حجم زیاد از داده های با تعداد ویژگی بالا، برای یادگی به منابع محاسباتی زیادی احتیاج دارد.
%%%%%%%%%%%%%%%%%%%%%%%%%%%%%%%%%%%%%%%%%%%%%%%%%%%%%%%%%%%%%%
\subsection{ماشین بولتزمن و مدل های تولید کننده}
یکی دیگر از روش های بسیار معروف یادگیری نظارت نشده، ماشین بولتزمن است. روش های نظارت نشده بررسی شده در این پایان نامه، نتوانستند صحت بسیار زیادی در تشخیص درهم تنیگی به دست آورند، و ممکن است بتوان با ماشین بولتزمن نتایج مطلوب تری به دست آورد. علاوه بر این، می توان از مدل های تولید کننده
\LTRfootnote{Generative Models}
استفاده کرد و حالت های در هم تنیده و جداپذیر برای یادگیری با روش های نظارت شده و شبه نظارت شده با این روش ها تولید کرد.‌
%%%%%%%%%%%%%%%%%%%%%%%%%%%%%%%%%%%%%%%%%%%%%%%%%%%%%%%%%%%%%%
\subsection{روش های دیگر یادگیری ماشینی}
معماری های بسیار زیادی برای یادگیری مسائل مختلف ارایه شده است، مانند معماری
\lr{Inception}
یا معماری
\lr{Dueling}،
روش های
\lr{Actor-Critcis}
مانند
\lr{A2C}
و
\lr{A3C}
و یا روش های جدید دیگر که
که ممکن است بتوانند نتایج قابل قبولی در حل مساله ی جداپذیری ارایه دهند.
%%%%%%%%%%%%%%%%%%%%%%%%%%%%%%%%%%%%%%%%%%%%%%%%%%%%%%%%%%%%%%
