\section{نتیجه گیری}
% 0.5
در این فصل ابتدا پارادوکس را به زبان ریاضی بیان کردیم و مساله را در حالت های مختلف بررسی کردیم. دیدیم که زمانی اطلاعات نابود می شود که ظرفیت کانال صفر شود. لذا ظرفیت کانال در حالت های مختلف و به خصوص در حالت های حدی کاملا بازتاب کننده و کاملا جذب کننده بررسی شد. پس از آن قسمت کلون شدن اطلاعات در پارادوکس بررسی شد و با محاسبه ی فیدلیتی سیاه چاله مشاهده شد که این کلون ها، کلون های تقریبی اند و همچنین دیدیم که نمی توان با استفاده از این کلون ها و پادکلون های ایجاد شده، اطلاعاتی را بازیابی کرد و لذا پارادوکسی در مورد کلون شدن اطلاعات شکل نمی گیرد.
