%%%%%%%%%%%%%%%%%%%%%%%%%%%%

\section*{چکیده}
\begin{spacing}{2}

درهم تنیدگی یکی از منابع کلیدی علوم اطلاعات کوانتومی است که شناسایی وضعیت های درهم تنیده را برای طیف گسترده ای از فناوری ها و پدیده های کوانتومی ضروری می کند. این مشکل به عنوان مساله جداپذیری نیز شناخته می شود و
NP-Hard
است.
در حالت ایده آل ، می‌خواهیم شاهدان بهینه بیابیم که با کمترین تعداد اندازه گیری ممکن، بتوانند حالت های در هم تنیده را تشخیص دهند.

در این پایان نامه، ما پیشنهاد می کنیم از شبکه های عصبی خود رمزگذار برای یافتن اندازه گیری های نیمه بهینه برای تشخیص درهم تنیدگی و ساخت شاهدان جدید درهم تنیدگی استفاده کنیم که می توانند حالت های درهم تنیده را با داده های ناکامل تشخیص دهند، به عنوان مثال فقط با سه اندازه گیری. شاهد شبکه عصبی، حالت ها را به عنوان ورودی می گیرد و در خروجی مشخص می کند که آیا حالت در خروجی درهم تنیده است یا خیر. در اینجا ما روش خود را روی سیستم دو کیوبیتی اعمال می کنیم، اما این روش می تواند برای سیستم هایی با ابعاد فضای هیلبرت بالاتر تعمیم یابد
.\\

\textbf{واژه‌های کلیدی:}
درهم تنیدگی، اطلاعات کوانتمی، مساله جداپذیری، یادگیری ماشین، هوش مصنوعی
\end{spacing}

\newpage\null\newpage
