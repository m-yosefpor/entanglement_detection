\section{تشخیص ناهنجاری}

در علم بررسی داده، به تشخیص حالت ها یا اتفاقات نادر که با غالب مشاهدات پیشین منطبق نیست، و مقدار زیادی تفاوت دارد، تشخیص ناهنجاری می گویند
\cite{TODO}.
این مساله کاربرد های بسیاری از جمله تشخیص تقلب (مثلا تقلب بانکی)، تشخیص نقص مقاومت سازه، مسائل پزشکی، خطاهای متنی و غیره دارد
\cite{TODO}.
دسته ای از تشخیص ناهنجاری با یادگیری ماشینی انجام می شود، که از آن ها می توان به تشخیص ناهنجاری نظارت نشده و تشخیص ناهنجاری شبه نظارت شده اشاره کرد.

در تشخیص ناهنجاری نظارت نشده، ماشین با فرض اینکه اکثر داده ها توزیع نرمال دارند، و فقط اندکی از آن ها خارج از این توزیع هستند، یک توزیع نرمال بهینه برای اکثریت داده ها پیدا میکند، و دسته ای که خارج از این توزیع هستند به عنوان ناهنجاری شناخته می شوند.

در روش توزیع ناهنجاری شبه بهینه، داده های یک کلاس به ماشین داده می شود، و ماشین با برازش کردن یک منحنی نرمال به صورت بهینه، آموزش داده می شود. در این حالت ماشین فقط با داده های یک کلاس آموزش داده شده است، و داده هایی که برچسب کلاس دیگر را دارند مشخص نیستند. به همین خاطر این یک روش شبه نظارت شده است. سپس داده های آزمایشی به ماشین داده می شوند و ماشین با توجه به اینکه این داده در کجای این توزیع قرار می گیرد، آن را به عنوان یک داده ی معمولی (هم برچسب با کلاس آموزش داده شده) یا به عنوان یک داده ی پرت یا ناهنجاری (غیر هم برچسب با کلاس آموزش داده شده) پیش بینی می کند.

در شکل
\ref{fig:anomaly_detection}
نحوه ی عملکرد این روش با مثالی آورده شده است.
همانگونه که در شکل
\ref{fig:anomaly1}
مشاهده می شود، دسته ای از داده های هم برچسب از یک کلاس، داده می شود. سپس ماشین با توجه به توزیع داده ها در فضا، یک توزیع نرمال برای این داده ها پیدا می کند. خطوط کانتور این توزیع در شکل
\ref{fig:anomaly2}
و نمودار سه بعدی آن در شکل
\ref{fig:guassian_dist}
آورده شده است. در نهایت اگر داده ای برای پیش بینی به ماشین داده شود، مقدار تابع گاوسی برای نقطه ی جدید محاسبه می شود و اگر از آستانه ای کمتر شد، این داده به عنوان یک ناهنجاری در نظر گرفته می شود
(
شکل
\ref{fig:anomaly3}
)


\input{\figurePATH{2}{anomaly_detection}}
