\section{یادگیری ماشین چیست}

یادگیری ماشین زیر مجموعه ای از هوش مصنوعی است که با استفاده از داده های ارایه شده توسط کاربر، به یادگیری مساله ای می پردازد.
از این روش معمولا برای تعیین خروجی مطلوب، با دادن ورودی مورد نظر استفاده می شود. تفاوت عمده ی آن با الگوریتم های کلاسیکی این است که در الگوریتم های کلاسیکی، برنامه نویس روش عملکرد ماشین بر داده های ورودی را طراحی و پیاده سازی می کند. اما در یادگیری ماشین، اینکه چگونه داده های ورودی را به خروجی مطلوب تبدیل کند، به ماشین داده نمی شود و خود ماشین طبق الگوریتیم و روش هایی، نحوه تبدیل ورودی به خروجی را یاد میگیرد
(
شکل
\ref{fig:classic_vs_ml}
\LTRfootnote{\tiny{https://medium.com/@artin.sinani/dont-be-intimidated-it-s-only-machine-learning-158f703ab387}}
)
.مثلا با دیدن مثال های آماده شده و یا دسته بندی داده ها با توجه به یک سری ویژگی مشترک.

\input{\figurePATH{2}{classic_vs_ml}}

الگوریتم های بسیار زیادی در یاگیری ماشین وجود دارد، که دسته ای از آن ها از روش های ریاضی به نسبت قدیمی تر استفاده می کنند و دسته ای از آن ها اخیرا ارایه شده و محبوبیت زیادی پیدا کرده اند که به شبکه های عصبی معروف هستند. این شبکه ها که معمولا به صورت لایه لایه هستند، می توانند تعداد اندکی لایه داشته باشند (که به آن ها یادگیری کم عمق می گویند) و یا تعداد زیادی لایه دارند (که به آن ها یادگیری عمیق گفته می شود).

یادگیری ماشین کاربرد های بسیار زیادی در همه ی علوم از جمله علوم کامپیوتر، علوم زیستی و بیولوژی، فیزیک، شیمی، زبان شناسی اقتصاد و غیره دارد. یکی از کاربرد های محبوب یادگیری ماشین، در علم داده است، که در آن با برسسی داده های ساختار یافته یا ساختار نیافته، به کسب اطلاعاتی از مساله پرداخته می شود.
در شکل
\ref{fig:ai_ml_deep}
ارتباط هوش مصنوعی، یادگیری ماشین، شبکه های عصبی، یادگیری عمیق و علم داده نشان داده شده است.

\input{\figurePATH{2}{ai_ml_deep}}
