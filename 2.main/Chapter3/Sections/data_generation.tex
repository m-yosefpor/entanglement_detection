\section{تولید داده ها}

برای این که بتوانیم از یادگیری ماشین استفاده کنیم، نیاز است چند میلیون حالت کوانتمی برای این کار استفاده شود.
ما این داده ها را به صورت مصنوعی، با انتخاب تصادفی از فضای هیلبرت سیستم تولید کردیم.
نحوه ی تولید داده ها با استفاده از کتابخانه ی
\lr{qutip}
در پایتون به این صورت است که
یک ماتریس هرمیتی
تصادفی
$H$
که
$4\times 4$
است
انتخاب می کنیم. سپس
$P=H^\dag H$
را محاسبه کرده و ماتریس نرمال شده ی
$N$
را به صورت
$N = \frac{P}{\Tr P}$
محاسبه می کنیم.
با استفاده از این ماتریس و تابع
\lr{Qobj}
کتابخانه ی
\lr{qutip}
یک ماتریس چگالی برای سیستم دوکیوبیتی تولید می کنیم.

پس از آنکه با این روش ۵ میلیون ماتریس چگالی تصادفی تولید کردیم، ویژگی های مورد نظر که همان
$\Gamma_{ij}$
ها بودند را توسط معادله ی
\ref{eqn:feature_cal}
محاسبه کردیم:

\[
   \Gamma =
  \left[ {\begin{array}{cccc}
\langle\sigma_x\otimes\sigma_x\rangle & \langle\sigma_x\otimes\sigma_y\rangle & \langle\sigma_x\otimes\sigma_z\rangle & \langle\sigma_x\otimes\ \mathbb{I}\rangle
\\ \langle\sigma_y\otimes\sigma_x\rangle & \langle\sigma_y\otimes\sigma_y\rangle & \langle\sigma_y\otimes\sigma_z\rangle & \langle\sigma_y\otimes\ \mathbb{I}\rangle
\\ \langle\sigma_z\otimes\sigma_x\rangle & \langle\sigma_z\otimes\sigma_y\rangle & \langle\sigma_z\otimes\sigma_z\rangle & \langle\sigma_z\otimes \mathbb{I}\rangle
\\ \langle\mathbb{I}\otimes\sigma_x\rangle & \langle\mathbb{I}\otimes\sigma_y\rangle & \langle\mathbb{I}\otimes\sigma_z\rangle &    \langle\mathbb{I}\otimes\ \mathbb{I}\rangle
  \end{array} } \right]
\]


سپس این ويژگی ها را به صورت یک سطر
$15 \times 1$
با حذف ۱۶ امین ویژگی بدیهی ۱ حاصل از
$ \mathbb{I} \otimes \mathbb{I}$
در آوردیم. این کار برای تمامی ۵ میلیون ماتریس چگالی انجام شد.

در قسمت اول این پروژه، از روش یادگیری نظارت شده برای تفکیک حالت ها استفاده کردیم. بنابرین در این روش ها نیاز است که برچسب
\LTRfootnote{Label}
هر یک از داده ها مشخص شود (که به معنی جداپذیر بودن یا در هم تنیده بودن است).
با توجه به اینکه در حالت دوکیوبیت این کار با معادله ی
\ref{eqn:augusiak}
و داشتن تمامی ۱۵ ویژگی امکان پذیر است، ما توانستیم تمامی داده ها را برچسب گذاری کنیم.
برچسب گذاری به صورت عدد ۰ برای جداپذیر ها و برچسب عدد ۱ برای درهم تنیده ها صورت گرفت.

تمامی این برچسب ها در ستون آخر (۱۶ ام) برای هر یک از ماتریس های چگالی ذخیره شد.
.پس داده نهایی به صورت یک ماتریس با ۵ میلیون سطر و ۱۶ ستون به دست آمد.
این داده ها را به صورت عمومی در
\lr{Gitlab}
بر روی اینترنت
\LTRfootnote{\small{\texttt{https://gitlab.com/quantum-machine-learning/autoencoder-entanglement-detection-optimizer}}}
قرار داده ایم و برای عموم قابل دسترس کرده ایم.
