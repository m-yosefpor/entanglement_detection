\section{معیار \lr{Peres – Horodecki} \lr{(PPT)}}





%%%%%%%%%%%%%%%%%%%%%%%%%
در حالتی که ابعاد زیر فضاهای هیلبرت سیستم مرکب، رابطه ی
$d_A\times d_B <= 6$
را براورده کنند، ( به طوری که
$ d_A $
و
$d_B$
ابعاد زیر فضاهای متناظر
\lr{A}
و
\lr{B}
باشد
)
یک شرط لازم و کافی برای جداپذیری به نام معیار
\lr{Peres-Horodecki}
یا
\lr{PPT}
ارایه شده است
\cite{horodecki_2009}.

در ابعاد بالاتر، که شرط ذکر شده برای ابعاد زیر فضا ها براورده نشود، این شرط فقط یک شرط لازم برای جداپذیری است و دیگر شرط کافی نیست.

این شرط بیان میکند که برای سیستم های دو بخشی، اگر حالت جداپذیر باشد، ترانهاده ی جزئی
\LTRfootnote{Partial Transpose}
آن مقادیر ویژه نامنفی دارد.
به عبارت دیگر، اگر ترانهاده جزئی یک سیستم دوبخشی نسبت به یکی از زیر سیستم هایش یک مقدار ویژه منفی داشته باشد، آن سیستم حتما در هم تنیده است.
این نتیجه مستقل از انتخاب زیر سیستم برای گرفتن ترانهاده جزئی است
\cite{horodecki_2009}.

به عنوان مثال ماتریس چگالی
$\rho$
که زیر مجموعه فضای
$\mathcal{H} = \mathcal{H}_A \otimes \mathcal{H}_B $
است را در نظر بگیرید.
این حالت را می توان به صورت زیر نمایش داد:

\input{\formulaPATH{1}{rho_bi}}

ترانهادی جزئی این سیستم نسبت به بخش
\lr{B}
به صورت زیر می شود:

\input{\formulaPATH{1}{rho_TB}}

در نتیجه برای یک حالت جداپذیر داریم:

\begin{equation}
\rho_{AB}^{T_B} = \sum_i p_i \rho_A^{(i)} \otimes {\rho_B^{T}}^{(i)}
\end{equation}

که این ماتریس مثبت معین است
\cite{horodecki_2009}.
بنابرین اگر
$\rho_{AB}^{T_B} < 0 $
باشد، حالت حتما در هم تنیده است.

برای سیستم های دو کیوبیتی معیار ساده تری بر اساس معیار
\lr{PPT}
توسط
\lr{Augusiak}
ارایه شده است
\cite{Augusiak2008}.
این معیار بر اساس دترمینان ماتریس ترانهاده جزئی است و یک شرط لازم و کافی است.

\input{\formulaPATH{1}{augusiak}}
