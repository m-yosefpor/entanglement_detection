\section{نمایش حالت های دوکیوبیت}





%%%%%%%%%%%%%%%%%%%%%%%%%
در حالتی که سیستم دو کیوبیتی است، یعنی از دو زیر سیستم با با بعد دو، تشکیل شده است، می توان حالت سیستم را به صورت زیر نمایش داد
\cite{chuang}:

\input{\formulaPATH{1}{rho_2}}

که در این رابطه
$\sigma_1, \sigma_2, \sigma_3$s are Pauli matrices and $\sigma_0$
ماتریس های پاولی هستند و
$\mathbb{I}$
نمایشگر عملگر همانی است.

اگر دو طرف این رابطه را در
$(\sigma_k \otimes \sigma_l)$
ضرب کنیم و سپس رد
\LTRfootnote{Trace}
ماتریس را محاسبه کنیم، داریم:

\input{\formulaPATH{1}{proof_gamma}}

در نتیجه مقادیر
$\Gamma_{ij}$
ها را می توان به صورت زیر به دست آورد:

\input{\formulaPATH{1}{feature_cal}}

همانطور که مشاهده می شود، این مقادیر
$\Gamma_{ij}$
چیزی جز مقدار چشم داشتی اندازه گیری های
$(\sigma_i \otimes \sigma_j)$
روی سیستم نیستند.

با توجه به اینکه مقدار چشم داشتی
$\mathbb{I} \otimes \mathbb{I}$
روی سیستم مقدار بدیهی
$\Gamma_{00} = 1$
را نتیجه می دهد، می توان با انجام
۱۵
اندازه گیری متناظر ضرب این ماتریس های پاولی، مضارب
$\Gamma_{ij}$
را به دست آورد و در نتیجه حالت یک سیستم دو کیوبیت را به طور کامل توصیف کرد.
با این وجود انجام
۱۵
اندازه گیری در آزمایشگاه هزینه و زمان بر است، و لازم است بتوانیم با تعداد کمتری آزمایش، حالت های در هم تنیده را تشخیص دهیم.

در این پایان نامه، ما با استفاده از روش های یادگیری ماشینی، روش هایی را ارایه می دهیم که بتوان بخش قابل توجه ای از حالت های در هم تنیده را با تعداد اندازه گیری های خیلی کمتر تشخیص دهیم.
