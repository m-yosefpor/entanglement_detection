\section{درهم تنیدگی}



%%%%%%%%%%%%
در سیستم های مرکب، پدیده ای فیزیکی به نام درهم تنیدگی
\LTRfootnote{Entanglement}
 ممکن است رخ بدهد به طوری که حالت فیزیک هر یک از این زیر سیستم ها را نمیتوان به صورت مستقل از زیر سیستم های دیگر توصیف کرد، و اندازه گیری های ویژگی های فیزیکی زیر سیستم ها با یکدیگر همبستگی دارد
\cite{Terhal2002}.
درهم تنیدگی منبع مهمی در علوم اطلاعات کوانتمی محسوب می شود
و همه ی سیستم های مرکب این خاصیت را ندارند
\cite{Cohen2008}.
سیستم کوانتمی که درهم تنیده نیست را جداپذیر می گویند.
درهم تنیدگی را می توان به صورت کمی درآورد و به روش هایی میزان درهم تنیدگی یک سیستم را به دست آورد
\cite{Vedral1997}.

برای یک سیستم دو بخشی
\LTRfootnote{bi-partite}،
فضای هیلبرت متناظر به صورت
$\mathcal{H} = \mathcal{H}_A \otimes \mathcal{H}_B$
است. در این حالت، یک ماتریس چگالی جداپذیر خوانده می شود اگر و تنها اگر بتوان آن را به صورت ترکیب محدبی
\LTRfootnote{convex combination}
از حالت هایی نوشت که هر یک به صورت ماتریس چگالی در هر یک از زیر سیستم ها باشند
\cite{Li2018}:

\input{\formulaPATH{1}{rho_sep}}

در غیر این صورت، به آن حالت درهم تنیده گفته می شود.
