\section{حالت و سیستم کوانتمی}
% def pure and mix


%%%%%%%%%%%
\subsection{حالت کوانتمی}
در تئوری اطلاعات کوانتمی، حالت یک سیستم فیزیکی، به صورت یک ماتریس چگالی
$\rho$
عضو فضای هیلبرت سیستم نشان داده می شود.
ماتریس چگالی، خواص زیر را دارد
\cite{chuang}:

\input{\formulaPATH{1}{rho_prop}}

%%%%%%%%%%%
\subsection{حالت کوانتمی سیستم مرکب}
حالت کوانتمی یک سیستم مرکب به صورت یک ماتریس چگالی عضو ضرب تنسوری فضاهای هیلبرت زیرسیستم های آن است.

\input{\formulaPATH{1}{rho_composit}}


%%%%%%%%%%
\subsection{حالت خالص و مخلوط}


یک حالت کوانتمی را می توان به صورت زیر نمایش داد
\cite{chuang}:

\begin{equation}\label{mixed_rep}
    \rho = \sum_{i}{p_i \ket{\psi_i} \bra{\psi_i}}
\end{equation}

به طوریکه
$\sum_{i}{p_i} = 1$.
این نمایش برای یک حالت کوانتمی یکتا نیست. در صورتی که بتوان نمایشی از ماتریس چگالی پیدا کرد به طوری که
فقط یکی از ضرایب غیر صفر (و قاعدتا مساوی ۱)
باشد، به این حالت، یک حالت خالص
\LTRfootnote{pure}
گفته می شود.
یک حالت خالص متناظر یک بردار
$\pk$
در فضای هیلبرت است. ماتریس چگالی این حالت به صورت زیر است:

\begin{equation}
\rho_{pure} = \pk \pb
\end{equation}


اگر حالت خالص نباشد، به آن یک حالت مخلوط
\LTRfootnote{mixed}
گفته می شود.

می توان نشان داد
\cite{chuang}
که در یک حالت خالص است اگر و تنها اگر
$\gamma = \Tr{\rho^2} = 1$
باشد.
از پارامتر
$\gamma$
به عنوان خلوص
\LTRfootnote{purity}
یاد میشود.
برای خلوص حالت کوانتمی داریم:

\begin{equation}
    \frac{1}{d} \leq \gamma \leq 1
\end{equation}

که در آن
$d$
بعد فضای هیلبرت است. همانطور که مشخص است، کران بالا زمانی رخ می دهد که حالت، خالص باشد. همچنین کران پایین زمانی رخ می دهد که ماتریس کاملا مخلوط باشد.
یک ماتریس کاملا مخلوط در نمایش
\ref{mixed_rep}
به صورت زیر است:

‍\begin{equation}
    \rho_{completely-mixed} = \frac{1}{d} \sum_{i=1}{d} \ket{\psi_i} \bra{\psi_i} = \frac{1}{d} I_d
\end{equation}


\subsection{تجزیه اشمیت}

در جبر خطی، تجزیه ی اشمیت نحوه ای از نمایش یک بردار در یک فضای هیلبرت تشکیل شده از ضرب تانسوری دو زیر فضا است. فرض کنید
$H_1$
و
$H_2$
دو فضای هیلبرت با ابعاد متناظر
$n$
و
$m$
باشند به طوریکه
$n \geq m$.
برای هر بردار
$w \in H_1 \otimes \H_2$
وجود دارد مجموعه های متعامد
$\{ u_1, u_2, ..., u_m\} \subset H_1$
و
$\{ v_1, v_2, ..., v_m\} \subset H_2$
به طوریکه:

\begin{equation}
    w = \sum_{i=1}^{m}{\alpha_i u_i \otimes \v_i}
\end{equation}

که در آن
$\alpha_i$
ها حقیقی، نامنفی و یکتا (قابل جابجایی) هستند
\cite{hoffman}.
