\section{تقارن استوانه ای}

حالت های کوانتمی که در آزمایشگاه استفاده می شوند،‌ تقارن های خاصی دارند که معمولا از تقارن های دستگاه تولید کننده ی آن ها ناشی می شود.
این تقارن ها، شرایطی را روی ماتریس های چکالی الزام می کنند که باعث می شود گستره ی آن ها خاص تر شود. در نتیجه ممکن است اطلاعات تقارن حالت ها، به تشخیص درهم تنیدگی کمک کند.  چون که تقارن ها، شرط های لازمی که در واقع روابطی بین ویژگی ها هستند را ایجاد می کنند، در این قسمت ما تقارن استوانه ای را مورد بررسی قرار داده ایم. در این جا محور تقارن استوانه ای
$z$
در نظر گرفته شده است.

\begin{equation}\label{eqn:sym_cond1}
(R_z(\theta) \otimes R_z(\theta))\> \rho\> (R_z(\theta)^\dag R_z(\theta)^\dag) = \rho
\end{equation}

برای یک ماتریس چگالی با تقارن استوانه ای، معادله ی
\ref{eqn:sym_cond1}
برقرار است. در نتیجه باید شرط زیر برای همهی زوایای
$\theta$
برقرا باشد:

\begin{equation}\label{eqn:sym_cond2}
    \forall \theta \enspace [R_z(\theta)\otimes R_z(\theta) , \rho] = 0
\end{equation}

که در آن عملگر دوران به این صورت است:

\begin{equation}\label{eqn:sym_cond3}
R_z(\theta) = e^{i\theta\sigma_z} \implies (\forall \theta \enspace [e^{i\theta\sigma_z}\otimes e^{i\theta\sigma_z} , \rho] = 0)
\end{equation}

با ساده سازی معادله ی
\ref{eqn:sym_cond3}
به رابطه ی زیر می رسیم:

\begin{equation}\label{eqn:sym_cond4}
[\sigma_z\otimes \mathbb{I} + \mathbb{I}\otimes \sigma_z , \rho] = 0
\end{equation}

Substituting $\rho = \frac{1}{4}\sum a_{ij} \sigma_i \otimes \sigma_j$ results in:
\begin{equation}
\sum a_{ij} [\sigma_z,\sigma_i]\otimes \sigma_j + \sum a_{ij} \sigma_i \otimes [\sigma_z,\sigma_j]=0
\end{equation}
using $[\sigma_i,\sigma_j] = 2i\epsilon_{ijk} \sigma_k$ we derive a condition for cylindrical symmetry:
\begin{equation}
\begin{split}
    2i \sum_j a_{xj}\sigma_y\otimes\sigma_j - a_{yj}\sigma_x\otimes\sigma_j \\
    + a_{jx}\sigma_j\otimes\sigma_y - a_{jy} \sigma_j\otimes\sigma_x = 0
\end{split}
\end{equation}

سپس ماتریس های چگالی را با الگوریتمی که در کد آمده است،‌ به صورتی پیدا می کنیم که تمام شرایط گفته شده در بالا را برآورده کند. سپس تمامی مدل های ارایه شده در قسمت های قبل را مجددا با این داده ها آموزش می دهیم. در این حالت نتایج بهتری از قبل به دست می آوریم. نتایج حاصل شده با وجود تقارن استوانه ای، در شکل
\ref{fig:acc_rec_n}
آورده شده است. همانگونه که مشاهده می شود، در این حالت، با ۴
اندازه گیری خطی، می توان تقریبا به تفکیک کامل دست یافت. همچنین فقط با یک اندازه گیری خطی به صحت
۹۰٪
دست یافتیم.
