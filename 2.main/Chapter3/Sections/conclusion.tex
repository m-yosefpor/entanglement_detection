\section{نتیجه گیری}

در این فصل ابتدا به نحوه ی تولید داده ها و بررسی آماری آن ها پرداختیم. سپس انواع مدل های یادگیری ماشینی کلاسیکی و شبکه عصبی روی داده ها با داشتن توصیف کامل سیستم بررسی شد و دیدیم که چندی از این مدل ها می توانند به تفکیک کامل دست یابند. سپس دو معماری جدید مبتنی بر شبکه های خود رمزگذار برای تشخیص درهم تنیدگی با تعداد آزمایش کمتر خطی و غیر خطی ارایه کردیم. همچنینی اثر نوفه بر این سیستم ها و اهمیت ویژگی ها برای تشخیص درهم تنیدگی را بررسی کردیم. در آخر به بررسی چند مدل یادگیری نظارت نشده و شبه نظارت شده پرداختیم. در فصل بعد، به جمع بندی نتایج و پیشنهادها برای ادامه ی کار می پردازیم.
