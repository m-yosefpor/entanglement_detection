\section{بررسی سایر رویکرد های موجود}

روش های متعددی برای تفکیک حالت های جداپذیر و درهم تنیده وجود دارد که از آن ها می توان به روش های کلاسیکی از جمله بازسازی توموگرافیکی
\LTRfootnote{Tomographic}
حالت های کوانتمی
\cite{Lu2016}،
معیار ماتریس کورایانس
\cite{Guhne2007}،
معیار همزمانی
\LTRfootnote{Concurrence}
\cite{Wootters1998,Rungta2001,DeVicente2007}،
عملگر های شاهد
\LTRfootnote{Witness Operators}
\cite{horodecki_1996,Terhal2000}،
و تفکیک با استفاده از تنظیمات جمعی خاص
\LTRfootnote{Collective settings}
\cite{Pezze2016}
اشاره کرد.
همچنین روش هایی مبتنی بر یادگیری ماشین نیز ارایه شده است
\cite{Wisniewska2015,Gao2018,Ma2018,Lu2018,Gray2018,Deng2018,Levine2018,Liu2018,Qiu2019}.
در برخی کار های دیگر روش هایی با استفاده از الگوریتم های قابل اجرا روی کامپیوتر های کوانتمی ارایه کرده اند
\cite{Cai2015,behrman2002}.

باید خاطر نشان کرد در تمامی روش های یادگیری ماشینی پیشین که در بالا ذکر شده اند، این تفکیک با داشتن توصیف کامل سیستم انجام شده است. به عنوان مثال برای حالت دوکیوبیت نیاز دارند که تمامی
۱۵
اندازه گیری انجام شود.
اما در روش پیشنهادی این پایان نامه فقط با استفاده از نتایج تعداد اندکی اندازه گیری این کار انجام می شود.

لازم به ذکر است که اگر توصیف کامل سیستم موجود نباشد، نمی توان حالت های تفکیک پذیر و در هم تنیده را به صورت
۱۰۰٪
کامل از یکدیگر تفکیک کرد، و فقط می توان قسمتی از آن ها را با یقین به صورت در هم تنیده یا جدا پذیر تشخیص داد. پس در واقع روش پیشنهادی این پایان نامه، شاهد های درهم تنیدگی برای تشخیص قابل توجه حالت های درهم تنیده را با تعداد کمتری اندازه گیری ارایه می دهد. اگرچه نشان می دهیم با همین روش با داشتن توصیف کامل سیستم،
(
یعنی هر ۱۵ اندازه گیری
)
می توان حالت های درهم تنیده و تفکیک پذیر را به صورت تقریبا کامل نیز از یکدیگر تفکیک کرد.
همچنین نشان می دهیم با تغییر آستانه
\LTRfootnote{Threshold}،
این روش می تواند معیار جداپذیری
\LTRfootnote{Separability Criterion}
با تعداد کمتری اندازه گیری نیز ارایه دهد.
