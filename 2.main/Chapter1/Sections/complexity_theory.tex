\section{پیچیدگی محاسباتی}


نظریه ی پیچیدگی محاسباتی، به سختی حل مساله های محاسباتی مبتنی بر میزان مصرف منابع می پردازد، و با توجه به این دسته بندی برای مسائل ارایه می دهد.
در این نظریه، مفاهیم مجردی از قبیل ماشین تورینگ
\LTRfootnote{Turing Machine}
بیان می شود که مدل ریاضی یک ماشین محاسباتی مانند کامپیوتر است.

ماشین های تورینگ انواع متفاوتی دارند که از جمله آن ها می توان به ماشین تورینگ قطعی
\LTRfootnote{Deterministic Turing Machine}،
ماشین تورینگ غیر قطعی
\LTRfootnote{Non-deterministic Turing Machine}،
ماشین تورینگ کوانتمی
\LTRfootnote{Quantum Turing Machine}،
ماشین تورینگ متقارن
 \LTRfootnote{Symmetric Turing Machine}
 و غیره اشاره کرد.

مسائلی را که بتوان با ماشین تورینگ قطعی،‌در زمان چندجمله ای
\LTRfootnote{Polynomial time}
حل کرد، در دسته مسائل
\lr{P}
قرار می دهند. همچنین مسائل که با ماشین تورینگ غیرقطعی در زمان چند جمله ای حل بشوند، در دسته مسائل
\lr{NP}
جای میگیرند. می توان نشان داد که همه ی مسائلی که در زمان چند جمله ای توسط ماشین تورینگ قطعی حل می شوند را می توان در زمان چند جمله ای با ماشین تورینگ غیر قطعی نیز حل کرد
\cite{computational_complexity}.
پس می توان نتیجه گرفت که دسته مسائل
\lr{P}
زیر مجموعه ای از مسائل
\lr{NP}
هستند. اما این سوال که آیا می توان همه ی مسائلی که توسط ماشین تورینگ غیر قطعی در زمان چند جمله ای حل می شوند را توسط ماشین تورینگ قطقی هم در زمان چند جمله ای حل کرد یا نه، هنوز یک سوال بی پاسخ است. هرچند با اینکه اثباتی برای این مساله که آیا
\lr{P}
$\neq$
\lr{NP}
یافت نشده است، همه ی شواهد نشان می دهد که چنین چیزی امکان پذیر نیست و تا کنون کسی موفق نشده است الگوریتمی برای این کار بر حسب ماشین تورینگ قطعی پیدا کند.


با توجه به اینکه کامپیوتر های کلاسیکی با ماشین تورینگ قطعی مدل می شوند، پس مسائلی که در دسته ی
\lr{NP}
جای میگیرند را نمی توان با آن ها در زمان چند جمله ای حل کرد. به عبارتی با بزرگ شدن اندازه ی مساله، تعداد عملیات محاسباتی لازم برای حل مساله، سریع تر از هر تابع چند جمله ای رشد می کند، و به سرعت، مساله غیر قابل حل می شود.

دسته ای از مسائل
\lr{NP}
وجود دارند، که از سخت ترین مسائل دسته ی
\lr{NP}
به شمار می روند. به عبارتی، هر مساله در دسته ی
\lr{NP}
را میتوان به آن مساله در زمان چند جمله ای کاهش کرد.
به این دسته از مسائل
\lr{NP-Complete}
گفته می شود.
همچنین هر مساله که از نظر سختی محاسباتی، حداقل به اندازه ی سخت ترین مسائل
\lr{NP}
پیچیدگی محاسباتی داشته باشد را
\lr{NP-Hard}
می گویند. ارتباط این دسته از مسائل در شکل
\ref{fig:P_NP_NP_Hard}
\LTRfootnote{\small{https://en.wikipedia.org/wiki/NP-completeness}}
نشان داده شده است.


\input{\figurePATH{1}{P_NP_NP_Hard}}
