\section{جمع بندی}
در این پایان نامه، قصد داشتیم تا پارادوکس اطلاعات سیاه چاله را با رویکرد تئوری اطلاعاتی حل کنیم. ابتدا در فصل اول به بررسی کیفی پارادوکس پرداختیم و گفتیم که با توجه به اینکه نباید اطلاعات در یک سیستم ایزوله از بین برود، و همچنین با توجه به اینکه  نمی توان اطلاعات کوانتمی را دقیقا کلون کرد، لذا این سوال مطرح می شود که اگر جسمی به سیاه چاله فروپاشی کند و تشکیل سیاه چاله بدهد یا پس از تشکیل سیاه چاله به درون آن فرستاده شود، آیا اطلاعات به صورت تابش به بیرون نشت می کند؟ اگر اطلاعات در تابش موجود نیست، پس به نظر می رسد که پس از تبخیر کامل سیاه چاله اطلاعات به کلی نابود شده است. اگر اطلاعات در داخل تابش موجود است، پس با در نظر گرفتن
\lr{nice-slice}
زمانی وجود دارد که اطلاعات هم در داخل سیاه چاله موجود است و هم در خارج آن؛ پس اطلاعات کلون شده است. لذا این در هر صورت به پارادوکس منجر می شود. 

برای حل این پارادوکس، تلاش هایی با روش های متعددی ارائه شده بود که روشی  که ما از آن استفاده کردیم، شیوه ی تئوری اطلاعاتی بود و اشکال پارادوکس را در تعریف دقیق مساله به زبان تئوری اطلاعاتی دانستیم. برای این که بتوانیم دقیقا این مساله را بررسی کنیم، ابتدا به ابزار های ریاضی تئوری اطلاعات کلاسیک و کوانتمی احتیاج داشتیم که آن ها را در فصل 2 معرفی کردیم.

در فصل 3 با استفاده از ابزار های مطرح شده در فصل 2، مسئله را به زبان دقیق ریاضی بیان کرده و ظرفیت کانال سیاه چاله در حالت های مختلف مورد بررسی قرار گرفت. با محاسبه ی ظرفیت کانال سیاه چاله در حالت های زود هنگام و دیر هنگام و همچنین برای تابش های هاوکینگ و تابش بر انگیخته، می توان سرنوشت اطلاعات را دنبال کنیم و به دو مشکل این  پارادوکس پاسخ دهیم. مشکل اول نابود شدن اطلاعات بود که باید نشان دهیم آن اطلاعات نابود نمی شود، مشکل دیگر، قضیه ی 
\lr{no-cloning}
بود. و باید نشان دهیم اطلاعات هم زمان در دو مکان مختلف وجود ندارد.


کانال سیاه چاله توسط تابش هاوکینگ یک کانال با ظرفیت کلاسیکی و همچنین کوانتمی صفر است 
($\chi(\cn) = 0 \, , \, Q(\cn) = 0$)
.
به عبارت دیگر یعنی بین این تابش و ورودی کانال، چه جسم پرتاب شده درون آن بعد از تشکیل و چه جسمی که قبل از تشکیل سیاه چاله در آن وجود داشته است ، هیچ ارتباطی وجود ندارد و این تابش صرفا یک تابش گرمایی می باشد. لذا نمی توان اطلاعات را در این تابش یافت. پس برای بررسی پارادوکس باید به تابش دیگر یعنی تابش بر انگیخته رجوع کرد و سرنوشت اطلاعات را آنجا جست و جو کرد.


سپس به بررسی تابش برانگیخته در مود های زود هنگام پرداختیم. در این حالت فرض کردیم که جسمی قبل از شکل گیری سیاه چاله در حالتی آماده شده است و پس از فروپاشی به سیاه چاله، اثری از آن جسم نمی توان مشاهده کرد به جز تابش های هاوکینگ و برانگیخته. تابش هاوکینگ که دارای اطلاعات نمی باشد چون دارای ظرفیت صفر است. اما برای تابش برانگیخته به دست آوردیم که در حالت کلاسیک و کوانتمی یک کانال 
\lr{Unruh}
است و در هر دو حالت دارای ظرفیتی غیر صفر  است.
پس می توان اطلاعات جسم اولیه را از این تابش بازسازی کرد و لذا اطلاعات نابود نشده است. همچنین چون چیزی به جز تابش برانگیخته و تابش هاوکینگ نداریم، و اطلاعاتی در تابش هاوکینگ وجود ندارد، پس این با قضیه ی 
\lr{no-cloning} 
در هماهنگی می باشد. و پارادوکس کاملا رفع می شود.



سپس به بررسی مود های دیر هنگام پرداختیم، یعنی زمانی که بعد از شکل گیری سیاه چاله جسمی به سمت آن پرتاب شود. ظرفیت کلاسیکی این نوع سیاه چاله را در حالت
$g'_k = g_k$
بررسی کردیم که منجر به ظرفیتی غیر صفر شد. برای محاسبه ی ظرفیت کوانتمی
این مسئله را در دوحالت حدی بررسی کردیم، ابتدا در حالت کاملا بازتاب کننده دیدیم که در این حالت یک کلون (منظور از کلون، کلون تقریبی می باشد)
به سمت خارج و یک پادکلون به سمت داخل ایجاد می شود. اثبات کردیم که کلون های ایجاد شده در خارج توسط تابش های بر انگیخته مجددا به کانال 
\lr{Unruh}
منجر می شود و لذا همان طور که گفته شد دارای ظرفیت غیر صفر است. پس می توان اطلاعات را از این تابش بازیابی کرد و هیچ اطلاعاتی نابود نمی شود. اما برای بررسی قضیه ی 
\lr{no-cloning}
باید علاوه بر تابش بدون ویژگی هاوکینگ، پاد کلون های فرستاده شده به داخل را هم در نظر بگیریم و ببینیم آیا اگر آلیس اطلاعاتی را از خارج سیاه چاله به سمت داخل سیاه چاله بفرستد، آیا باب که داخل سیاه چاله هست می تواند اطلاعات را بازیابی کند یا خیر. در این حالت نشان دادیم که کانال تشکیل شده مکمل کانال 
\lr{Unruh}
می باشد و لذا این کانال یک کانال در هم تندیگی شکننده است و در نتیجه ظرفیت آن صفر می باشد. پس یعنی باب نمیتواند اطلاعات را از این پادکلون ها بازیابی کند و هیچ اطلاعاتی داخل سیاه چاله وجود ندارد. پس اطلاعات فقط خارج سیاه چاله بازتابیده شده است و قضیه ی 
\lr{no-cloning}
در این حالت نیز برقرار می باشد و پارادوکسی وجود ندارد.




در حالت حدی بعدی، سیاه چاله را کاملا جذب کننده در نظر گرفتیم و مشاهده کردیم که در این حالت اگر آلیس اطلاعاتی را به سمت سیاه چاله بفرستد و باب تمامی تابش های حاصل از سیاه چاله را دریافت کند، با توچه به اینکه کانال کوانتمی حاصل، یک کانال 
\lr{depolarizing}
می باشد، ظرفیت آن صفر است و نمی تواند هیچ اطلاعاتی را دریافت کند. اگرچه در این حالت دیدیم که اطلاعات را نمی توان بازیافت کرد، اما سوال اصل باقی می ماند که پس از تبخیر کامل سیاه چاله چه اتفاقی می افتد و آیا می توان از آنچه داخل آن بوده است اطلاعات را بازیابی کرد یا خیر.

اگر چه در حالت مود های دیرهنگام فقط دو حالت حدی کاملا بازتاب کننده و کاملا جذب کننده را بررسی کردیم، اما انتظار می رود که در حالت های دیگر نیز این نتایج برقرار باشد
\cite{qit}
.

در بخش آخر فصل 3 نیز، به این قضیه پرداختیم که با اینکه سیاه چاله ها از ذرات پرتاب شده به سمت آن کلون ایجاد می کنند، اما این مساله با قضیه ی 
\lr{no-cloning}
در تناقض نیست، چون کلون های ایجاد شده کلون های تقریبی می باشند و همیشه وفاداری آن ها از حالت وفاداری بهیته کمتر می باشد و در حالت حدی کاملا بازتاب کننده به مقدار بیشینه وفاداری خود که همان وفاداری بهینه است می رسد.
