\section{انواع یادگیری ماشین}

در یادگیری ماشین، نیاز به تعداد نسبتا زیادی داده ی آموزشی از ویژگی ها است. سپس با توجه به این داده ها، میخواهیم مساله دسته بندی یا رگرسیون را انجام دهیم. در ادامه به سه دسته کلی این الگوریتم ها با توجه به برچسب گذاری داده ها می پردازیم.

\subsection{یادگیری نظارت شده}
اگر دسته بندی این داده های آزمایش، از قبل به نوعی مشخص شده باشد، یا به عبارتی همه ی آن ها برچسب گذاری شده باشند، و الگوریتم ماشین از این برچسب ها برای یادگیری استفاده کنند، به این روش یادگیری نظارت شده می گویند. تعدادی از این روش ها که در این پایان نامه استفاده شده اند عبارتند از:
\lr{Ridge, Random Forest, Support Vector, Bagging, Gradient Boosting}
و غیره. همچنین دسته ای از شبکه های عصبی مانند
\lr{Feed Forward Neural Network (FNN)}،
\lr{Convolutional Neural Network (CNN)}،
\lr{Recursive Neural Network (RNN)}
نیز از دسته ی یادگیری نظارت شده هستند.

\subsection{یادگیری نظارت نشده}
اما اگر الگوریتم برای دسته بندی، فقط از ویژگی ها استفاده کند و استفاده ای از برچسب داده ها نکند، به آن ها یادگیری نظارت نشده می گویند. تعدادی از این روش ها که در این پایان نامه استفاده شده اند عبارتند از:
\lr{OPTICS},
\lr{K-means},
\lr{DBSCAN},
\lr{Birch}.
همچنین از روش های شبکه عصبی نظارت نشده می توان به
\lr{Boltzman Machine}
ها اشاره کرد، که دسته ای مهم از روش های یادگیری نظارت نشده هستند.

\subsection{یادگیری شبه نظارت شده}
دسته ای از روش ها هم هستند که ترکیب این دو اند. یعنی قسمتی از داده ها برچسب گذاری شده اند،‌و قسمت دیگری برچسب گذاری نشده اند.
این روش ها با استفاده از برچسب های موجود، سعی در بهتر کردن نتیجه دارند. روش های شبه نظارت شده ی مورد استفاده در این پایان نامه عبارتند از
\lr{LabelPropagation}،
\lr{LabelSpreading}،
و
\lr{Anomaly Detection}.
در ادامه مبانی تعدادی از این روش ها به اختصار توضیح داده می شود.
