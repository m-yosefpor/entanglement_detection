\section{ساختار پایان نامه و علامت گذاری}





%%%%%%%%%%%%%%%%%%%%%%%%%
در این پایان نامه، ابتدا به مفاهیم یادگیری ماشینی مورد نیاز برای بررسی این مسئله در فصل ۲ پرداخته می شود. این فصل به یادگیری  مفاهیم ضروری یادگیری ماشین و شبکه های عصبی و روش های بررسی نتایج آن ها می پردازد.
سپس در فصل سوم ابتدا به توضیح نحوه تولید داده ها و بررسی های آماری آن ها می پردازیم. در ادامه سعی می کنیم با مدل های یادگیری ماشین کلاسیک و شبکه های عصبی، ابتدا با داشتن توصیف کامل سیستم به تفکیک تقریبا کامل حالت های جداپذیر و در هم تنیده برسیم. سپس مدل های جدیدی برای تشخیص حالت های درهم تنیده با تعداد کمتری آزمایش غیر خطی، و روش دیگر با تعداد کمتری آزمایش خطی، ارایه کرده و نتایج هر یک را بررسی می کنیم. پس از آن با فرض تقارن های آزمایشگاهی، مدل را مجدد آموزش می دهیم و نتایج بهبود یافته ای به دست می آوریم.
پس از آن، تعداد از مدل های یادگیری نظارت نشده و شبه نظارت شده را بررسی می کنیم.
.در آخر نیز به پایداری سیستم و اثر نویز های مخرب بر آن می پردازیم
.در فصل آخر نیز، به جمع بندی مطلب می پردازیم و پیشنهادات موجود برای ادامه کار تبیین می شود.

همچنین در این پایان نامه، عملگر ها و ماتریس ها به جز ماتریس های چگالی با حروف بزرگ نشان داده شده اند  (به طور مثال
$U$
نمایانگر یک عملگر و
$\rho$
نمایانگر یک ماتریس چگالی می باشد
) و
برای بردار ها نمایش برا-کتی دیراک (به عنوان مثال
$\pk$
) در نظر گرفته شده است.
