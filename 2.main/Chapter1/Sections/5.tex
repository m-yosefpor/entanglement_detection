\section{شاهد درهم تنیدگی}


شاهد درهم تنیدگی
\LTRfootnote{Entanglement witness}،
معیاری برای تشخیص حالت های درهم تنیده است. این شاهد ها در واقع یک شرط کافی برای درهم تنیدگی هستند،
اما باید توجه کرد یک شاهد درهم تنیدگی، لزوما شرط لازمی برای درهم تنیده بودن نیست.
بنابرین اگر شروط شاهد درهم تنیدگی برآورده نشود، نمی توان نتیجه گرفت که حالت جداپذیر است.
\cite{Terhal2002}،
ولی اگر شروط شاهد برآورده شود، می توان نتیجه گرفت که حالت حتما درهم تنیده است.

به زبان ریاضی، برای هر حالت درهم تنیده
$\rho$،
یک عملگر هرمیتی
$w$
وجود دارد به طوری که
$\Tr(w\rho) < 0$
و
$\Tr(w\sigma) \geq 0$
برای هر حالت جداپذیر
$\sigma$
\cite{horodecki_1996}.
به این عملگر
$w$
یک شاهد درهم تنیدگی گفته میشود، چرا که هر گاه
$\Tr(w\rho) < 0$
شود، قطعا این حالت درهم تنیده است.

\input{\figurePATH{1}{witness}}
 فرص کنید که فضای حالت های کوانتمی مطابق شکل
 \ref{fig:witness}
 باشد. آنگاه یک شاهد درهم تنیدگی فضای حالت ها را به دو زیر فضا افراز میکند، که یک بخش فقط شامل حالت های درهم تنیده، و بخش دیگر شامل حالت های جداپذیر و درهم تنیده است.
 پس با هر شاهد درهم تنیدگی، ممکن است فقط بخشی از حالت های درهم تنیده تشخیص داده شوند.
 در حالتی که شاهد درهم تنیدگی، یک شرط لازم و کافی باشد، شاهد فضا را به دقیقا دو زیر فضای در هم تنیده و جداپذیر افراز میکند، یا به عبارتی می تواند همه ی حالت های درهم تنیده را تشخیص دهد
 (
 شکل
 \ref{fig:iff_witness}
 ).
