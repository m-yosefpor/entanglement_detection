\section{پایداری و اثر نوفه}

نتایج اندازه گیری های موجود در آزمایشگاه، همواره همراه با نوفه مزاحم هستند که از محیط اطراف یا حتی خود دستگاه اندازه گیری ناشی می شوند.
بنابرین باید مطمئن شویم نتایج به دست آمده در بخش قبل، تا حد قابل قبولی در مقابل نوفه مقاوم هستند. چراکه اگر با وجود اندکی نوفه، و تغییر اندک پارامتر های ورودی، صحت شاهد ارایه شده توسط شبکه عصبی به طرز قابل توجهی افت کند، این مدل ها در عمل خیلی کاربردی نیستند.

در این قسمت، ما نوفه جمع شونده با توزیع گاوسی، که از نوفه های معمول آزمایشگاه هست را بررسی کرده ایم. نوفه به صورت یک عدد تصادفی با توزیع گاوسی تولید شد، و با ویژگی های موجود جمع می شود. میانگین این توزیع را صفر در نظر گرفته، و همچنین این کار را با انحراف از معیار های متفاوت برای این توزیع تکرار کرده ایم. سپس در هر مرحله صحت شبکه عصبی را مجددا آزمایش کرده، و نتایج را در شکل
\ref{fig:noise}
نشان داده ایم.

\input{\figurePATH{3}{noise}}

همانطوری که مشاهده می شود، صحت شبکه عصبی به طور آرامی با افزایش نوفه افت می کند، و افت ناگهانی دیده نمی شود. به طور مثال با اضافه کردن نوفه با انحراف معیار
۰٫۱،
صحت فقط حدود ۴ درصد افت میکند و همچنان بالای
۹۰٪
باقی می ماند(داده های مساله همگی بین
$-1$
و
$1$
هستند).
